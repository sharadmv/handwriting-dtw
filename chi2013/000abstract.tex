This paper presents a new form of user input for writing. Using existing commercial computer vision devices, precise and realtime 3D finger position data is now readily available. With this in mind, this paper proposes a mechanism by which users can write "in the air" and characters be recognized in real time. In order to do this, a new dataset needs to be created, partitioned into candidate time series, specifically the characters in the alphabet, and testing time series, which are words that to be recognized. To construct this dataset, the LEAP Motion, a commercial computer vision device, will be used to record and store data. The experiment will consist collecting the same data from  (insert number) people to account for differences in handwriting. This paper will also go into an approach to actually determine which letters were written in the data time series. This approach uses the dynamic time warping algorithm as a distance metric for a nearest neighbour similarity search, uing the candidate time series as queries for the larger data time series, using a series of optimizations to make the search feasible in realtime.
