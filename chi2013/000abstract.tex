This paper presents a new form of user input for writing. Using existing commercial computer vision devices, precise and realtime 3D finger position data is now readily available. With this in mind, we can imagine users writing "in the air" and aim to recognize the characters in realtime. In order to do this, we need to construct a new dataset, partitioned into candidate time series (the characters in the alphabet) and testing time series, which are words that we aim to recognize. To construct this dataset, we are using the LEAP Motion, a commercial computer vision device, to record and store data. The experiment will consist collecting the same data from  (insert number) people to account for differences in handwriting. This paper will also go into an approach to actually determine which letters were written in the data time series. This approach uses the dynamic time warping algorithm as a distance metric for a nearest neighbour similarity search. We are using the canditate time series as queries for the larger data time series, using a series of optimizations to make the search feasible in real time. 
