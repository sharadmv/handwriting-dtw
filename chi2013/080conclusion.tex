This paper presents a new type of user input for writing: given finger data from the LEAP Motion device, identify characters and words that are written in the air. This problem is novel because no pen up/pen down gesture exists that determines the beginning and end of data. Rather, characters must be recognized in real time. We propose a data-based similarity search algorithm using dynamic time warping and its recent optimizations to do some simple matching. Future work will include extending the recognition algorithm to arbitrary gestures and the use of the LEAP Motion in different user scenarios than handwriting recognition. These include using a web browser, listening to music, and a replacement to the mouse and keyboard altogether. For example, users can use their computer as normal by moving their finger as the mouse. When a text input area is selected by the mouse, the handwriting input mode would be used, and the stream of finger data points would be interpreted as letters and sent as input to the computer. This process will require the use of 3D data and will thus increase the complexity of the problem. However, such a 3D gesture system would enable us to use more complex types of input, such as American Sign Language and would help disabled people interact with computers much more easily.\\
