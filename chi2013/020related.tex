Existing online handwriting recognition techniques depend on a pen up/pen down gesture to window the input data. Essentially, there is a known beginning and end to user input. This is not the case with this paper. We are using an input device that constantly streams the location of the fingers within its field of view so this type of gesture is not as easily done. Another technique used is the segmentation of the data points. This is difficult as it is hard to determine the end and beginning of segments, so typically unsupervised learning and data-driven approaches are used~\cite{plamondon2000online}. The statistical approaches to this problem use Hidden Markov Models or use a combination of HMMs and neural networks to recognize characters~\cite{plotz2009markov}. Hilbert Warping has been proposed as an alignment method for handwriting recognition~\cite{ishida2010hilbert}. Other scenarios have been proposed, including one where an LED pen is tracked in the air. This allows for 3D data to be interpreted, but also allows for the beginning and end of input to be clearly defined~\cite{asano2010visual}. Finally, treating the handwriting problem like speech recognition, ie treating the input points as a signal, allows in place algorithms with handwriting feature vectors to be used, but the same problem of segmentation arises~\cite{starner1994online}. They also have problems with accuracy in identification. 
Another area of application of these techniques is sketch recognition, or digitizing drawings. The methods typically involve searching for sketch primitives and then combining them, but also rely on pen up/pen down gestures~\cite{hammond2011recognizing}. 
