Traditional character recognition technology is widely applied to such problems as converting scanned books to text and converting images of bank checks into valid payments. These problems can be divided into offline and online recognition, which can be considered Optical Character Recognition (OCR) and Intelligent Character Recognition (ICR) respectively. 

The task of identifying characters in a time series requires data to test on. Therefore, a new dataset needs to be created, partitioned into candidate time series, specifically the characters in the alphabet, and testing time series, which are words that to be recognized. To construct this dataset, the LEAP Motion, a commercial computer vision device, will be used to record and store data. The experiment will consist collecting the same data from  (insert number) people to account for differences in handwriting.
